%-------------------------
% Resume in Latex
% Author : Aras Gungore
% License : MIT
%------------------------

\documentclass[letterpaper,11pt]{article}

\usepackage{latexsym}
\usepackage[empty]{fullpage}
\usepackage{titlesec}
\usepackage{marvosym}
\usepackage[usenames,dvipsnames]{color}
\usepackage{verbatim}
\usepackage{enumitem}
\usepackage[hidelinks]{hyperref}
\usepackage{fancyhdr}
\usepackage[english]{babel}
\usepackage{tabularx}
\usepackage{hyphenat}
\usepackage{fontawesome}
\input{glyphtounicode}


%---------- FONT OPTIONS ----------
% sans-serif
% \usepackage[sfdefault]{FiraSans}
% \usepackage[sfdefault]{roboto}
% \usepackage[sfdefault]{noto-sans}
% \usepackage[default]{sourcesanspro}

% serif
% \usepackage{CormorantGaramond}
% \usepackage{charter}


\pagestyle{fancy}
\fancyhf{} % clear all header and footer fields
\fancyfoot{}
\renewcommand{\headrulewidth}{0pt}
\renewcommand{\footrulewidth}{0pt}

% Adjust margins
\addtolength{\oddsidemargin}{-0.5in}
\addtolength{\evensidemargin}{-0.5in}
\addtolength{\textwidth}{1in}
\addtolength{\topmargin}{-.5in}
\addtolength{\textheight}{1.0in}

\urlstyle{same}

\raggedbottom
\raggedright
\setlength{\tabcolsep}{0in}

% Sections formatting
\titleformat{\section}{
  \vspace{-4pt}\scshape\raggedright\large
}{}{0em}{}[\color{black}\titlerule \vspace{-5pt}]

% Ensure that generate pdf is machine readable/ATS parsable
\pdfgentounicode=1

%-------------------------
% Custom commands

\newcommand{\resumeItem}[1]{
  \item\small{
    {#1 \vspace{-2pt}}
  }
}


\newcommand{\resumeSubheading}[4]{
  \vspace{-2pt}\item
    \begin{tabular*}{0.97\textwidth}[t]{l@{\extracolsep{\fill}}r}
      \textbf{#1} & #2 \\
      \textit{\small#3} & \textit{\small #4} \\
    \end{tabular*}\vspace{-7pt}
}


\newcommand{\resumeSubSubheading}[2]{
    \vspace{-2pt}\item
    \begin{tabular*}{0.97\textwidth}{l@{\extracolsep{\fill}}r}
      \textit{\small#1} & \textit{\small #2} \\
    \end{tabular*}\vspace{-7pt}
}


\newcommand{\resumeEducationHeading}[6]{
  \vspace{-2pt}\item
    \begin{tabular*}{0.97\textwidth}[t]{l@{\extracolsep{\fill}}r}
      \textbf{#1} & #2 \\
      \textit{\small#3} & \textit{\small #4} \\
      \textit{\small#5} & \textit{\small #6} \\
    \end{tabular*}\vspace{-5pt}
}


\newcommand{\resumeProjectHeading}[2]{
    \vspace{-2pt}\item
    \begin{tabular*}{0.97\textwidth}{l@{\extracolsep{\fill}}r}
      \small#1 & #2 \\
    \end{tabular*}\vspace{-7pt}
}


\newcommand{\resumeOrganizationHeading}[4]{
  \vspace{-2pt}\item
    \begin{tabular*}{0.97\textwidth}[t]{l@{\extracolsep{\fill}}r}
      \textbf{#1} & \textit{\small #2} \\
      \textit{\small#3}
    \end{tabular*}\vspace{-7pt}
}

\newcommand{\resumeSubItem}[1]{\resumeItem{#1}\vspace{-4pt}}

\renewcommand\labelitemii{$\vcenter{\hbox{\tiny$\bullet$}}$}

\newcommand{\resumeSubHeadingListStart}{\begin{itemize}[leftmargin=0.15in, label={}]}
\newcommand{\resumeSubHeadingListEnd}{\end{itemize}}
\newcommand{\resumeItemListStart}{\begin{itemize}}
\newcommand{\resumeItemListEnd}{\end{itemize}\vspace{-5pt}}

%-------------------------------------------
%%%%%%  RESUME STARTS HERE  %%%%%%%%%%%%%%%%%%%%%%%%%%%%


\begin{document}

%---------- HEADING ----------

\begin{center}
    \textbf{\Huge \scshape Richard Antony Cruz Rojas} \\ \vspace{3pt}
    \small
    \faMobile \hspace{.5pt} \href{tel:51931062761}{+51 931 062 761}
    $|$
    \faAt \hspace{.5pt} \href{mailto:cruz.rojas.ra@gmail.com}{cruz.rojas.ra@gmail.com}
    $|$
    \faLinkedinSquare \hspace{.5pt} \href{https://www.linkedin.com/in/richard-antony-cruz-rojas/}{LinkedIn}
    $|$
    \faGithub \hspace{.5pt} \href{https://github.com/ricktonycr}{GitHub}
    $|$
    %\faGlobe \hspace{.5pt} \href{https://arasgungore.github.io}{Portfolio}
    %$|$
    \faMapMarker \hspace{.5pt} \href{https://www.google.com/maps/place/Arequipa/@-16.4040524,-71.5390115,14z/data=!3m1!4b1!4m6!3m5!1s0x91424a487785b9b3:0xa3c4a612b9942036!8m2!3d-16.4090474!4d-71.537451!16zL20vMDFweTg3?entry=ttu}{Arequipa, Perú}
\end{center}



%----------- EDUCATION -----------

\section{Educación}
	\vspace{3pt}
		\resumeSubHeadingListStart
			\resumeEducationHeading
				{Universidad Nacional de San Agustín}{Arequipa, Perú}
				{Ingeniería de Sistemas;  \textbf{Quinto superior} \textbf{GPA: 14.52/20.00}}{Marzo 2013 \textbf{--} Diciembre 2018}
				{}{}
		\resumeSubHeadingListEnd

%----------- SKILLS -----------

\section{Skills}
  \vspace{2pt}
  \resumeSubHeadingListStart
    \small{\item{
        \textbf{Idiomas hablados:}{Español (avanzado), Inglés (intermedio)} \\ \vspace{3pt}
        \textbf{Lenguajes:}{ Java, Genexus, Javascript, Typescript, SQL, Bash, C/C++, Python, Go} \\ \vspace{3pt}
        \textbf{Tecnologías:}{ Git, Spring Framework, Tomcat, ANTLR, Jenkins, Node.js, React.js, MySQL, MongoDB, Docker, Podman, GTK, AWS, GCP, OpenCV, PyTorch, TensorFlow, WebRTC, WebGL, Trhee.js, OpenGL.} \\ \vspace{3pt}
        \textbf{Metodologías:}{ Agile, Scrum, OOP, Functional Programming, DevOps, CI/CD, TDD} \\ \vspace{3pt}
    }}
  \resumeSubHeadingListEnd

%----------- EXPERIENCE -----------

\section{Experiencia}
  \vspace{3pt}
  \resumeSubHeadingListStart

    \resumeSubheading
      {Red Social Universitaria Uconecta}{Arequipa, Perú}
      {Practicante Ingeniería de Sistemas / Programador}{Marzo 2013 \textbf{--} Marzo 2015, Part-time}
        \resumeItemListStart
            \resumeItem{Se hizo un análisis de la interfaz de la página, proponiendo cambios especficos siguiendo lineamientos de usabilidad, diseño y experiencia de usuario. Se implementó la propuesta utilizando CSS, HTML y PHP, según los lineamientos del framework Drupal.}
            \resumeItem{Se investigó y usó el framework Drupal para el desarrollo de módulos de interacción entre usuarios, tales como reacciones y comentarios en posts para la red social.}
            \resumeItem{Se desarrollo el módulo de comunicación con el API de Facebook.com para la autenticación y compartir la página de la red social entre sus amigos de Facebook.}
        \resumeItemListEnd

    \resumeSubheading
      {Centro de Investigación CITESOFT}{Arequipa, Perú}
      {Asistente de Investigación}{Julio 2016 \textbf{--} Marzo 2018, Part-time}
        \resumeItemListStart
            \resumeItem{Worked on the “Arçelik Digital Home Energy” project in a collaborative effort with DAI-Labor at the Technical University of Berlin under the supervision of \href{https://www.linkedin.com/in/prof-dr-dr-h-c-sahin-albayrak-65452a1/}{\color{blue}Prof. Dr. Şahin Albayrak}.}
            \resumeItem{Simulated discovery, pairing, and data exchange processes using the EEBUS protocol suite with C\# and Go. Migrated the framework from Go to C++ in order to ensure future adaptability for smart home IoT devices.}
            \resumeItem{Implemented the TLS protocol for secure data exchange using the X.509 standard and integrated multicast DNS for seamless communication to complement the development of SHIP and SPINE protocols.}
        \resumeItemListEnd

    % \resumeSubheading
      % {Scale AI}{San Francisco, California, United States (Remote)}
      % {Prompt Engineer}{Mar 2023 \textbf{--} Sep 2023, Freelance}
        % \resumeItemListStart
            % \resumeItem{Crafted elaborate Turkish prompts for various generative AI tasks. Tailored prompts using well-defined formats such as natural language questions and structured inputs based on the task's domain to fine-tune the LLM, improving the chatbot's responses and enhancing overall performance.}
            % \resumeItem{Provided user feedback by reviewing and scoring the chatbot's responses to ensure they align with intended chatbot behaviors and the chatbot produces accurate and contextually appropriate responses to given prompts.}
            % \resumeItem{Engaged in collaborative meetings with cross-functional teams and project coordinators, actively seeking guidance, addressing queries, and collectively brainstorming strategies to generate higher-quality prompts.}
        % \resumeItemListEnd

    \resumeSubheading
      {Max Planck Institute for Intelligent Systems}{Stuttgart, Baden\textbf{-}Württemberg, Germany}
      {Undergraduate Researcher}{Jun 2022 \textbf{--} Aug 2022, Internship}
        \resumeItemListStart
            \resumeItem{Engaged in collaborative research within the Robotics, Collectives and Learning subgroup at the Physical Intelligence Department with Ph.D. students \href{https://www.linkedin.com/in/sinan-ozgun-demir-981311129/}{\color{blue}Sinan Özgün Demir} and \href{https://www.linkedin.com/in/alpkaracakol/}{\color{blue}Alp Can Karacakol} on a project about 3D printing and heat-assisted magnetic programming of soft machines under the supervision of \href{https://www.linkedin.com/in/metin-sitti-0a8a712/}{\color{blue}Prof. Dr. Metin Sitti}.}
            \resumeItem{Optimized a C++ ROS package for real-time conversion of 3D motion controller events to ROS messages, achieving high-frequency and buffer-free synchronization.}
            \resumeItem{Developed an Arduino Mega driver to activate a laser and pressure regulator, for monitoring coil temperatures and PID tuning to control the coil currents. Established a robust ROS-Arduino communication network by integrating ROSSerial and handshaking to simulate a 3D printing and magnetic programming process with Python.}
            % \resumeItem{Designed the project's system and software architecture, flowchart, and state machine diagram. Expanded G-code capabilities by incorporating spherical coordinates to specify the direction of magnetization for shape morphing.}
        \resumeItemListEnd
    
    % \resumeSubheading
      % {Nanonetworking Research Group, Boğaziçi University}{Istanbul, Turkey}
      % {Undergraduate Researcher}{Oct 2021 \textbf{--} Jun 2022, Part-time}
        % \resumeItemListStart
            % \resumeItem{Worked on the “Design and Implementation of Molecular Communication Systems Using Index Modulation” project under the supervision of \href{https://www.linkedin.com/in/alipusane/}{\color{blue}Prof. Dr. Ali Emre Pusane}.}
            % \resumeItem{Simulated the Brownian motion of molecules in a SISO MCvD system and predicted simulation parameters such as receiver radius, diffusion coefficient, and transmitter-receiver distance using CNNs with Keras and TensorFlow.}
            % \resumeItem{Plotted the arrival of molecules per symbol duration in a SISO MCvD system using Binomial, Poisson, and Gaussian model approximations with MATLAB.}
            % \resumeItem{Ran Monte Carlo simulations of the Gaussian model to encode and decode randomized binary sequences in a SISO MCvD system using BCSK modulation technique and calculated the bit error rate on Z-channel.}
        % \resumeItemListEnd

    % \vspace{20pt}
    \resumeSubheading
      {SESTEK Speech Enabled Software Technologies}{Istanbul, Turkey}
      {AI Research and Development Intern}{Jan 2022 \textbf{--} Feb 2022, Internship}
        \resumeItemListStart
            % \resumeItem{Executed diverse NLP tasks, including NER, POS tagging, sentiment analysis, text classification, and extractive and generative QA using transformers and Hugging Face libraries. Conducted a thorough literature review on information retrieval and reading comprehension to stay updated on SOTA ML models.}
            \resumeItem{Developed a generative QA system with dense passage retrieval (DPR) and retrieval-augmented generation (RAG) techniques using the Haystack framework and PyTorch.}
            \resumeItem{Worked on a Turkish open-domain QA system made by fine-tuning BERTurk and XLM-Roberta models. Tabularized exact match and F1 scores derived from DeepMind's XQuAD and various Turkish data sets.}
        \resumeItemListEnd

    % \resumeSubheading
      % {Meteksan Defense Industry Inc.}{Ankara, Turkey}
      % {Analog Design Engineering Intern}{Jul 2021 \textbf{--} Aug 2021, Internship}
        % \resumeItemListStart
            % \resumeItem{Designed and simulated numerous analog circuits such as voltage-mode controlled buck converter, phase-shifted full-bridge isolated DC-DC converter, and EMI filters with LTspice. Integrated these circuits and implemented a 320 W power distribution unit to be used in a radar system's power circuit board.}
            % \resumeItem{Researched real-world compatible electronic components to be used in such circuits including GaNFETs, high-side gate drivers, and Schottky diodes.}
            % \resumeItem{Assembled PCBs of both common and differential mode filters and used VNA Bode 100 to measure the cut-off frequencies.}
        % \resumeItemListEnd
    
  \resumeSubHeadingListEnd



%----------- AWARDS & ACHIEVEMENTS -----------

\section{Awards \& Achievements}
  \vspace{2pt}
  \resumeSubHeadingListStart
    \small{\item{
        \textbf{High Honors Degree:}{ Awarded to Bachelor alumni who have graduated with a GPA greater than or equal to 3.50 by Boğaziçi University. (Jul 2023)} \\ \vspace{3pt}

        \textbf{TÜBİTAK 2247-C Intern Researcher Scholarship:}{ Awarded to students who take part in research projects carried out by the Scientific and Technological Research Council of Turkey (TÜBİTAK). (Dec 2021 \textbf{--} Jun 2022)} \\ \vspace{3pt}
    
        % \textbf{National University Admission Exam (YKS):}{ Ranked $75^{th}$ in Mathematics and Science among ca. 2.3 million candidates with a test score of 489.92/500. (Jul 2018)} \\ \vspace{3pt}
        
        \textbf{KYK Outstanding Success Scholarship:}{ Awarded to students who have been ranked in the top 100 on National University Admission Exam by Higher Education Credit and Hostels Institution (KYK). (Sep 2018 \textbf{--} Jun 2023)} \\ \vspace{3pt}
        
        % \textbf{Boğaziçi University Success Scholarship:}{ Awarded to students who have been ranked in the top 100 on National University Admission Exam by Boğaziçi University. (Sep 2018 \textbf{--} Jun 2023)} \\ \vspace{3pt}
        
        \textbf{Kocaeli Science High School Valedictorian Award:}{ Graduated as the highest ranked student. (Jun 2018)}
    }}
  \resumeSubHeadingListEnd



%----------- PROJECTS -----------

\section{Projects}
    \vspace{3pt}
    \resumeSubHeadingListStart
      
      \resumeProjectHeading
        {\textbf{Filters and Fractals} $|$ \emph{\href{https://github.com/arasgungore/filters-and-fractals}{\color{blue}GitHub}}}{}
          \resumeItemListStart
            \resumeItem{A C project which implements a variety of image processing operations that manipulate the size, filter, brightness, contrast, saturation, and other properties of PPM images from scratch and recursive fractal generation functions to model popular fractals including Mandelbrot set, Julia set, Koch curve, Barnsley fern, and Sierpinski triangle.}
          \resumeItemListEnd
      
      \resumeProjectHeading
        {\textbf{Chess Bot} $|$ \emph{\href{https://github.com/arasgungore/chess-bot}{\color{blue}GitHub}}}{}
          \resumeItemListStart
            \resumeItem{A C++ project in which you can play chess against an AI with a specified decision tree depth that uses alpha-beta pruning algorithm to predict the optimal move. Aside from basic moves, this mini chess engine also implements chess rules such as castling, en passant, fifty-move rule, threefold repetition, and pawn promotion.}
          \resumeItemListEnd
      
      \resumeProjectHeading
        {\textbf{DS\&A Projects} $|$ \emph{\href{https://github.com/arasgungore/CMPE250-projects}{\color{blue}GitHub}}}{}
          \resumeItemListStart
            \resumeItem{Five Java projects that apply DS\&A concepts such as discrete-event simulation using priority queues, Dijkstra's shortest path algorithm, Prim's algorithm to find the minimum spanning tree, Dinic's algorithm for maximum flow problems, and weighted job scheduling with dynamic programming to real-world problems.}
          \resumeItemListEnd
      
    \resumeSubHeadingListEnd



%----------- RELEVANT COURSEWORK -----------

% \section{Relevant Coursework}
  % \vspace{2pt}
  % \resumeSubHeadingListStart
    % \small{\item{
        % \textbf{Major coursework:}{ Materials Science, Electrical Circuits I-II, Digital System Design, Numerical Methods, Probability Theory, Electronics I-II, Signals and Systems, Electromagnetic Field Theory, Energy Conversion, System Dynamics and Control, Communication Engineering, Pattern Recognition, Introduction to Digital Signal Processing, Introduction to Digital Communications, Introduction to Database Systems, Introduction to Image Processing, Machine Vision} \\ \vspace{3pt}
        
        % \textbf{Minor coursework:}{ Discrete Computational Structures, Introduction to Object-Oriented Programming, Data Structures and Algorithms, Computer Organization, Fundamentals of Software Engineering}
    % }}
  % \resumeSubHeadingListEnd



%----------- CERTIFICATES -----------

% \section{Certificates}
  % \resumeSubHeadingListStart
    
    % \resumeOrganizationHeading
      % {Procter \& Gamble VIA Certificate Program}{Feb 2022}{Project Management and Personal Productivity}
    
  % \resumeSubHeadingListEnd



%----------- ORGANIZATIONS -----------

% \section{Organizations}
  % \resumeSubHeadingListStart
    
    % \resumeOrganizationHeading
      % {Institute of Electrical and Electronics Engineers (IEEE)}{Feb 2022 -- Present}{Student Member}
    
  % \resumeSubHeadingListEnd



%----------- HOBBIES -----------

% \section{Hobbies}
  % \resumeSubHeadingListStart
    % \small{\item{Basketball, Swimming, Fitness, Eight-ball, Horology}}
  % \resumeSubHeadingListEnd



%----------- REFERENCES -----------

% \section{References}
  % \vspace{2pt}
  % \resumeSubHeadingListStart
    % \item{References available upon request.}
  % \resumeSubHeadingListEnd



%-------------------------------------------
\end{document}
